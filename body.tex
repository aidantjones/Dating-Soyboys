\pagestyle{fancy}
\fancyhf{}
\fancyhead[LE,RO]{Aidan Jones}
\fancyhead[RE,LO]{Dating Soyboys}
\fancyfoot[LE,RO]{\thepage}

\twocolumn

\section{Literature Review}

\subsection{Masculinities}

\subsubsection{Essentialist}

Masculinities scholar R.W. Connell summarizes essentialist masculinity with such traits as ``risk-taking, responsibility, irresponsibility, aggression'' and general toughness. However, she also notes that a defining aspect of essentialist masculinity is its ``arbitrary'' nature \hyperlink{connell}{(Connell 2005)}. There is a general, related framework of masculinity that individuals can construct from the above traits. But there is not a reliable source that from which one can derive all these traits. Essentialist masculinity, supposedly rooted in biology, is amorphous, in actuality.

For veg* diets, this manifests itself by drawing on the supposed biological necessity of meat-eating in humans and men, in particular. Evidence suggests that meat comprised a significant---but not total---portion of human and proto-human diets. Moreover, men were instrumental in providing it for their communities, but, again, did not dominate the hunting and calorie-sourcing market (\hyperlink{kaplan}{Kaplan et al. 2000}; \hyperlink{zink}{Zink and Lieberman 2016}). Despite the lack of proto-human sustinence patriarchy, many Americans since the \nth{19} century have deemed plant-based cultures as more effeminate than ``meat-eating Englishmen'' using this folklore \hyperlink{gambert}{(Gambert and Linné 2018)}. The story implicates that for strength and masculinity, one has to eat meat of the supposedly natural amount to which the white, American male adheres.

\subsubsection{Normative}

Some attempt to circumvent the unstable foundations of essentialist masculinity by relying on normative masculinity. This concept dictates what men should be and usually relies on essentialist concepts. It prescribes that although men are not entirely essentialist, it insists they should aspire to be \hyperlink{connell}{(Connell 2005)}.

Normativity usually takes the form of the Protein Myth in veg* contexts, the belief that plants cannot provide the necessary amount and types of protein required to remain healthy \hyperlink{woodvine}{(Woodvine 2009)}. This misconception is more gender neutral than some other concepts tied to the meat and masculinity complex. However, the gender normative viewpoint that men should be hunters because of their physiological adaptations necessitates that they need a source of animal protein for stamina and maintaining muscle. The Protein Myth aids in this regard when it contends that plant sources inadequately provide protein. The fallacy keeps veg* curious men from trying out a more plant-based lifestyle due to the supposed protein requirements hardwired into their male DNA.

\subsubsection{Semiotic}

Finally, there is semiotic masculinity, defining masculinity by which it is not: femininity. As a framework of analysis, has benefits over the former two methods. It does not intend to assume and reinforce essentialist masculinities. Rather, it observes things as they are and categorizes them based on the gender of their original actor \hyperlink{connell}{(Connell 2005)}.

More recently, Kristen Barber utilized a framework in her work on the men's grooming industry she dubs the ``Passive Voice.'' In this work, she studies the concept of masculinity indirectly. She asks women in the industry how they relate to the concept and how they think of the concept themselves \hyperlink{barber2}{(Barber 2019)}. This makes sense in the context of the study; much of the interactions in these spaces are with female groomers. It applies to the current research as well. I develop a conception of masculinity from the words of women. These women used implicit and explicit versions of semiotic masculinity during our discussions.

Plant-based lifestyles are often associated with feminine-coded dieting. Women are more aware of dieting and healthy foods which tend to be more plant-based \hyperlink{arganini}{(Arganini et al. 2012)}. Thus, the public relates plant-based lifestyles and foods with femininity. By extension, they connect masculine eating to unhealthy, meat-heavy meals. Cultural products like \emph{Seinfeld}'s ``The Wink'' succinctly encapsulate these concepts. While on a date with a woman at a steakhouse, Jerry finds its offerings too rich for his liking. He opts for a salad to the disdain of his date. She orders a hearty steak dinner which masculinizes her in contrast to Jerry. This decision is later admonished by his female confidant Elaine as well. In essence, choosing the light, vegetable-based salad emasculates Jerry \hyperlink{ackerman}{(Ackerman 1995)}.

\subsubsection{Performative}

Judith Butler's conception of sex and gender relates to semiotic masculinity in that it is not based on anything concrete. In fact, Butler argues that both sex and gender are a social construction. Thus, there are no static versions of the two. She argues gender only exists while being performed and thus is always in flux \hyperlink{butler}{(Butler 1990)}.

\paragraph{Precarious Manhood}

Precarious Manhood is a concept that has its basis in Butler's performative theory. It states that manhood has to be constantly earned by the approval of others. Thus, men harbor anxiety over having to continually prove themselves in this manner \hyperlink{vandello}{(Vandello and Bosson 2013)}.

For men, eating meat bolsters Precarious Manhood multiple times a day in front of others. However, where this manhood is especially apparent is when men cook. Cooking for men usually takes the form of grilling in social settings. In this environment, they maintain their masculine status by grilling and eating meat while performing these acts in front of an audience. Conversely, society views cooking as part of female nature. Thus, women's countless more hours cooking in the kitchen are not as recognized as when men cook on special occasions.

Works like Hollows's discuss this phenomenon. Celebrity chef Jamie Oliver's show \emph{The Naked Chef} tends to reconstruct cooking as a masculine, leisure activity. This interpretation differentiates it from the necessary, laborious cooking that many women have to thanklessly endure on a daily basis. Shows like these glorify and masculinize cooking in the public light, assuaging Precarious Manhood in the process \hyperlink{hollows}{(Hollows 2003)}

\subsubsection{Hegemonic}

Connell defines hegemonic masculinity as practices and structures that ensure the patriarchy. It may not always be violent. But hegemonic masculinity is often connected to types of institutional power and has the ability to dictate culture and behavior on a broad scale \hyperlink{connell}{(Connell 2017)}.

For meat, this takes the form of those high up in the meat marketing, processing, and developmental fields. The Meat Industry Hall of Fame is an organization that bestows honors upon those with significant contributions to the meat industry. Only three of the 92 people recognized are women \hyperlink{2019a}{(2019a)}.

But it also takes more unexpected forms. Notably, influential political leaders often have financial and symbolic ties to meat. Ted Cruz, pork product in hand, advertised for the National Pork Board at the Iowa State Fair while debating a woman on LGBT rights. His commanding stance at the grill heightened his dismissive, dominant attitude toward the debate \hyperlink{cbs}{(CBS News 2015)}. And according to Politico, the board takes unfairly from hog farmers. The money goes to Super-Pac-esque groups that benefit politicians \hyperlink{vinik}{(Vinik 2019)}.

Furthermore, President Trump has twice presented a spread of meat-laden fast-food-fare to champion football teams. He noted that these men ``aren't into salads'' and would want this ``great American food'' instead (\hyperlink{boren}{Boren and Bieler 2019}; \hyperlink{klein}{Klein 2019}). The president presenting this feast of processed meats reinforces this hegemony at a federal level by itself. However, he also does this in front of men playing an extremely masculine sport in America. This incident supports Trump's Precarious Masculinity and augments the symbolic masculine power of the meat.

\subsubsection{Men Foregoing Meat}

Men who choose not to consume meat risk losing some of their masculine status in the process. Thomas \hyperlink{thomas}{(2015)} demonstrated this impact in her study where participants rated profiles. Some of these profiles were of individuals following veg* lifestyles. She found that it was not always the veg* diet itself that caused emasculation. Rather, it is the \emph{choice} to follow it. This rejection aligns with masculinities portrayed in works like \emph{Guyland}. Men who do not look masculine can perform masculinity (e.g., rape culture and sports) so that male peers may accept them \hyperlink{kimmel}{(Kimmel 2008)}. Meat is yet another way to perform this same type of masculinity.

Ruby and Heine's \hyperlink{ruby}{(2011)} study was similar in that participants also rated profiles of vegetarians and omnivores. This study was unique, however, in that it controlled for the healthiness of the diets presented. Subjects ranked the vegetarian profiles as ``more virtuous.'' But they were still perceived as less masculine than their omnivorous counterparts.

A survey distributed to undergraduates at a Kentucky university by Rothgerber \hyperlink{rothgerber}{(2012)} found that men justified eating animals in a direct way. Women tended to try to dissociate meat from its animal origins. Men tended to vindicate meat consumption as a way to maintain their masculinity \hyperlink{rothgerber}{(Rothgerber 2012)}.

Brady and Ventresca \hyperlink{brady}{(2014)} analyzed media attention after prominent NFL running back Arian Foster announced his conversion to veganism. They explored the extent to which the media reinforced the link between football, hegemonic masculinity, and the consumption of meat \hyperlink{brady}{(Brady and Ventresca 2014)}.

One of the most recent entrants into the literature was conducted by Mycek \hyperlink{mycek}{(2018)}. She conducted in-depth interviews with veg* men about how they aligned their masculinity with their emasculating diet. Many of the subjects justified their diets using logic and science, even when the reasoning is usually coded as emotional (e.g., concern for the treatment of animals). This masculine rationality, the author argues, contrasts with the feminine-coded emotional reasoning that often accompanies women's explanations for a veg* diet. For example, a woman might cite an emotional connection with animals as her reason for being veg*. A man, instead, would cite research that proves animals feel pain on an equal level to humans.

\paragraph{Meat as a Symbol}

Meat as a symbol of masculinity has chiefly arisen from sociobiological musings. As discussed earlier, this symbolism has primarily stemmed from essentialist and normative forms of masculinity, propagating a caveman mythos. This narrative argues that because men were the primary providers of meat and calories for themselves and their communities, they need meat for strength (\hyperlink{kaplan}{Kaplan et al. 2000}; \hyperlink{zink}{Zink and Lieberman 2016}). Of course, this rarely persists in contemporary society where only a select few hunt to survive. Precarious Manhood has aggravated this essentialist form of masculinity into a \textit{social} phenomenon that cements the link between meat and masculinity.

One of the original and most seminal works on the subject is Carol Adams's \emph{The Sexual Politics of Meat}. Adams \hyperlink{adams}{([1990] 2010)} asserts that Western culture dictates ``men need meat,'' that it is a symbol of masculinity and strength, and that eschewing meat is a marker of femininity. As masculinity studies have grown over the past few decades, accepted definitions of masculinity have lined up with Adams' original assessment. Since the original publishing, other studies have confirmed Adams's observations. Sobal \hyperlink{sobal}{(2006)}, for instance, argued that the ways men eat and provide meat symbolize strength, health, and wealth depending on the situation.

\paragraph{Metrosexuality}
Metrosexuality is an intermediary between more traditional masculinities and the soyboys concept discussed below. This masculinity still emphasizes heterosexual desire. However, it also adds narcissistic tendencies with the goal of admiration by both men and women. The latter point usually requires men's participation in feminine-coded behaviors, such as cooking, cleaning, grooming, and healthy eating (\hyperlink{barber1}{Barber 2016}; \hyperlink{buerkle}{Buerkle 2009}; \hyperlink{simpson}{Simpson 2002}).

Individuals and corporations invested in maintaining hegemonic forms of masculinity have caught onto this trend. Buerkle \hyperlink{buerkle}{(2009)} discusses how, in an attempt to snuff out burgeoning metrosexuality, prominent users of meat, like burger chains, advertise their products as masculinity
reinforcers for their consumers. There are attempts to reject
metrosexual ``bluefish from cedar planks'' for ``burgers from wax-paper wrappers'' \hyperlink{buerkle}{(Buerkle 2009)}. Because ``fish meat is practically a vegetable,'' according to the masculine archetype character Ron Swanson on the TV series \emph{Parks and Recreation}, this feminine symbol is denied in favor of its cow-sourced counterparts \hyperlink{yang}{(Yang 2011)}.

\paragraph{Soyboys}

Recent discourse about plant-based diets and emasculation has taken the form of the insult ``soyboy,'' a new term at the time of this writing; as such, this study is one of the first to analyze the term. Google Trends reports that it became popular in late 2017 \hyperlink{google}{(Google 2019)}. Thus, history of the term is limited mostly to internet sources, confirmed by Gambert and Linné \hyperlink{gambert}{(2018)}. Urban Dictionary definitions describe Soyboys as a new way for those on the alt-right to refer to effeminate, feminist, liberal, Nintendo-loving men \hyperlink{2019b}{(2019b)}. This has started to build upon and replace the alt-right's historically used "cuck," short for cuckold. Beyond its original definition of a husband staying with a wife who cheats, it also refers to a passive man and implies someone with progressive views \hyperlink{oxford}{(Oxford University Press 2019)}.

The only academic source discussing the topic I found at the time of this writing is one by Gambert and Linné \hyperlink{gambert}{(2018)}. They trace the history of soy boy back from an emasculated rice eater stereotype. This prejudice was used as an excuse for colonialism in Asia during the \nth{19} century. Authoritative figures in the West at that time believed that eating animals made one intellectually and physically superior to cultures that had more plant-based diets \hyperlink{gambert}{(Gambert and Linné 2018)}.

Media portrayals of the term tend to focus on members of the alt-right who consume milk to retaliate against the consumption of emasculating milk alternatives supposedly drunk by soyboys (\hyperlink{hathaway}{Hathaway 2017}; \hyperlink{hosie}{Hosie 2017}). Gambert and Linné suggest these men draw upon the mythos surrounding milk in early-\nth{20} century America. Due to influence by large dairy corporations at the time, milk symbolized strength, colonialism, and idealized masculinity \hyperlink{gambert}{(2018)}.

\paragraph{Eco-fascism}

Another related concept stems from the idea of ``Blood and Soil,'' a Nazi ideology that advocates for conservationism, organic farming, and, often, a veg* diet based on German's believed sovereignty to their homeland. These people also link the perceived purity of veg* diets to the purity of the white race \hyperlink{forchtner}{(Forchtner and Tominc 2017)}. In fact, both eyewitness accounts and biomedical evidence suggest that Hitler was mostly vegetarian (\hyperlink{nikkhah}{Nikkhah 201}; \hyperlink{p}{P. et al. 2018}).

Some contemporary followers of this movement have adopted Hitler's stance by also following a more plant-based lifestyle. One website, \href{https://www.aryanism.net}{aryanism.net}, has an entire section lauding veganism \hyperlink{aryanism}{(Aryanism 2009)}. Another article describes how organic farming has become desirable by some members of the German right \hyperlink{mcgrane}{(McGrane 2013)}. One intriguing study discussed how the contents of the vegan, German, Neo-Nazi YouTube cooking show \textit{Balaclava Küche} (Balaclava Kitchen) intertwined Neo-Nazi ideology into the principles behind veg* diets \hyperlink{forchtner}{(Forchtner and Tominc 2017)}.

Further research between the link between Neo-Nazi and white supremacist ideology, their masculine ideals, and veg* diets is required and is out of the scope of this paper. However, it is interesting to consider how these concepts may start to be more influential as the alt-right continues to gain international traction. The link to prominent male figures in white supremacist circles such as Hitler could generate a new version of plant-based masculinity and end up challenging the soyboy concept.

\subsubsection{Summary}

When most think of the link between meat and masculinity, they either consider essentialist arguments for masculinity or the media's influence on the concept. However, as presented in this section, there are more than just two versions that tie the two ideas together. There are often subtle means by which meat and masculinity symbiotically interact with one another. This connection harmfully reflects upon men who decide not to associate with this symbol. Both men and women use the masculine symbolism and imagery associated with meat through various types of masculinity to police men's gender identity.

\subsection[Heterosexual and Bisexual Female Attraction]{Heterosexual and Bisexual\\ Female Attraction}

Using data from a major online dating website, Hitsch, Hortaçsu, and Ariely \hyperlink{hitsch1}{(2005)} found that women tend to prefer men who are in masculine, health-related lines of work and who have higher levels of education. A controlled study of speed dating by Fisman et al. \hyperlink{fisman}{(2006)} confirmed these results, adding that women also seek physical fitness and financial stability as well. A preferences survey given to 36 cultures around the world verifies the results above, confirming that high social status is desirable too \hyperlink{shackelford}{(Shackelford, Schmitt, and Buss 2005)}.

Americans tend to associate veg* diets with female, attractive,
intelligent, middle- and upper-class individuals. This often takes the form of the celebrity in media representation of veg* diets (\hyperlink{mooney}{Mooney and Lorenz 1997}; \hyperlink{mycek}{Mycek 2018}; \hyperlink{ruby}{Ruby and Heine 2011}). The American public tends to perceive veg* men as less attractive than their omnivore counterparts because of the emasculation associated with these diets. Mainstream American media publications such as \emph{The New York Times}, \emph{The Atlantic}, and \emph{NPR} have all reported on this phenomenon (\hyperlink{beck}{Beck 2018}; \hyperlink{brubach}{Brubach 2008}; \hyperlink{rothman}{Rothman 2018}; \hyperlink{ulaby}{Ulaby 2014}). This trend is even present in countries around the world, such as Argentina, Australia, China, Finland, Japan, The Netherlands, and Turkey (\hyperlink{delassio-parson}{DeLessio-Parson 2017}; \hyperlink{lockie}{Lockie and Collie 1999}; \hyperlink{morioka}{Morioka 2013}; \hyperlink{nath}{Nath 2011}; \hyperlink{roos}{Roos, Prättälä, and Koski 2001}; \hyperlink{schosler}{Schösler et al. 2015}).

In particular, Japan has the concept \emph{soushouku-kei danshi}. This term literally means ``grass-eating men'' and is called ``Herbivore Men'' in English. Herbivore Men are undesirable, feminized men who are not aggressive in pursuing romantic relationships and sex, have a more feminine fashion sense, and sometimes eat a less carnivorous diet. The idea became popular in both the Japanese and Western media (\hyperlink{charlebois}{Charlebois 2013}; \hyperlink{chen}{Chen 2012}; \hyperlink{harney}{Harney 2015}; \hyperlink{khan}{Khan 2016}; \hyperlink{lim}{Lim 2009}; \hyperlink{morioka}{Morioka 2013}; \hyperlink{neill}{Neill 2009}; \hyperlink{nicolae}{Nicolae 2014}; \hyperlink{otagaki}{Otagaki 2009}).

\subsection{Current Research}

Although many discuss the masculinity of veg* men, researchers tend to remain hypothetical. They will rely on rating fictional profiles, for example. It's only recently with studies such as Mycek's \hyperlink{mycek}{(2018)} that have started to explore the reality of living as a veg* man. No one has asked women, the primary judge in these situations for heterosexual veg* men, how they view their attractiveness and masculinity until now.

\section{Methods}

\subsection{Survey}

I utilize a multi-methods approach. First, I distributed a survey (provided in \hyperlink{appendix-a}{Appendix A}) via the snowball method. I circulated it at a veg* conference in Los Angeles along with my Facebook and Instagram social networks, instructing recipients to pass along the survey as well. The ad I used for both in-person and online contexts can be found in \hyperlink{appendix-c}{Appendix C}. 31 people responded to the survey from September 2018 to January 2019. I excluded three from the sample because they did not meet the study criteria, resulting in a sample size of 28. I used Google Forms for distribution and analyzed data with Stata/SE 15.1.

I recruited both men and women for the survey part of this analysis. I chose both to draw comparisons between the two genders and expand my network for potential interview subjects.

There are a couple of questions at the beginning that screen out anyone who has not been in a relationship with a vegan or vegetarian person for at least a month. I chose a month arbitrarily to make sure that interviewees would have had ample romantic experience with their partner for richer, detailed, and reliable responses. This did not end up mattering much as the average relationship length was 2.6 years after eliminating one outlier.

Furthermore, anyone who identifies as any other gender other than male or female was also eliminated. I am concerned with constructs of masculinity and femininity. Thus, the experience of genderqueer persons would not be applicable to the study.

The end of the survey asks basic demographic questions. A space was provided for women interested in being interviewed to leave contact information.

The bulk of the survey measures the gender expression of both the respondent and their ideal partner. I used a variety of methods to measure gender. First, I decided to forego more traditional measures of masculinity and femininity, like the defunct Bem Sex-Role Inventory (\hyperlink{bem}{Bem 1974}; \hyperlink{stenberg}{Stenberg 2017}). In fact, the only items that significantly correlated with men and women on the BSRI were the terms ``masculine'' and ``feminine.''

Recent research has discovered that a more explicit approach is more accurate as well \hyperlink{kachel}{(Kachel, Steffens, and Niedlich's 2016)}. Kachel et al.'s inventory asks about gender-roles, identities, interests, beliefs, and so on using a Likert scale of ``Very feminine'' to ``Very masculine.''

Comparing multiple inventories, including Bem's \hyperlink{bem}{(1974)} and Kachel et al.'s \hyperlink{kachel}{(2016)}, I found the latter's Traditional Masculinity and Femininity scale to be the most accurate in
both implicit and explicit gender reporting. For example, the Traditional Masculinity and Femininity scale was one of the only ones to distinguish in gender expression between both heterosexual and homosexual women and men. However, some other methods of measuring were also included that also draw upon dating preferences research, such as ones by \hyperlink{hitsch1}{Hitsch et al. (2005)} and \hyperlink{fisman}{Fisman et al. (2006)}.

Finally, I gathered demographic information in the final part of the survey, including age, education, and race. A copy of the survey is in  \hyperlink{appendix-b}{Appendix B}.

\subsection{Interview}

I conducted six in-depth qualitative interviews with women. Half were drawn from the survey sample and half were drawn from personal contacts or veg* social events at which I networked. All participants had the same screening criteria as the survey respondents. Subjects signed consent forms, were interviewed either in-person or over the phone, and transcripts were created from recordings. Interviews averaged approximately an hour long.

The average age of interviewees was 27 years old, the minimum being 21 and max 35. Three subjects identified as white with one specifying Armenian. Two more identified as Asian while the last identified as Latina. The sample is half heterosexual and half bisexual. All participants were either middle or upper class, save for one who identified with lower class. All interviewees were also vegan or vegetarian except for one who used to be.

I conducted deductive coding on transcripts using NVivo 12. I use a combination of versus and values coding to analyze the interviews. I utilize versus coding because I am analyzing the binary of masculinity and femininity. It was helpful to use this when first going through the texts to have some frame of reference for initial codings. I also employ value coding because my goal is to measure what these women believe about masculinity.

After an initial coding, I refined the codes and placed them into categories. The largest one is ``Strength'' for masculinity and ``Emotional'' for femininity. I added a third main code, ``Strong-Kindness,'' later that is the crux of the analysis below.

\subsection{Ethical Concerns}

After I reached out to all potential subjects from my survey pool, I destroyed all contact information. I gave interviewees pseudonyms and removed identifying details---such as location data and names, and other information as requested by the interviewee---from the transcripts. Interview audio was destroyed after transcription.

The rest of the data from this project is kept on an external, password-protected 256-bit AES encrypted drive. The only physical data from this research are consent forms. The original is scanned, stored on the encrypted drive, and then destroyed.

\subsection{Discussion}

After completing the collection phase, I discovered the qualitative part of the research answered the research question most directly. To uncover the interactions between the various variables at work for such a personal topic, such as emasculation, positive veg* qualities, and gender roles, interviewing is optimal choice. In principle, quantitative methods can approximate the quality of data achieved through qualitative ones, especially on a larger scale. But quantitative methods are not as appropriate in this context due to the small sample size.

However, novel qualitative research such as this can inform future quantitative data collection \hyperlink{pearce}{(Pearce 2002)}. In fact, a unique finding from the present survey data that may apply to this situation is discussed in the \hyperlink{future-research}{``Future Research''} section. Furthermore, the survey data provided a solid foundation for the qualitative findings.

\section{Results}
\subsection{Survey Descriptive Statistics}

\hyperlink{appendix-a}{Table 1 in Appendix A} provides detailed descriptive statistics of survey respondents. The sample consists of 15 women and 12 men. The average age is 26 years old. Two-thirds of the sample are heterosexual and a third are bisexual. Three-fourths of respondents are white. There are three Latinos and the rest are either Asian or multiracial. The average survey-taker is a college graduate and has an annual salary of \$50,000 to \$75,000. 2.6 years is the average relationship length with one outlier excluded. The demographics of these survey-takers align well with extant samples and veg* stereotypes: white, educated, middle-class women.

\subsection{Veg* Rationale Trifecta}

The reasons for going veg* that participants in this study detailed tended to fall into three categories: environment, ethical, and health. One said, "I like the way {[}veganism{]} makes me feel physically. And I also support animal rights." Another said she "supported everything: environmentally, health-wise, and also for animals." Yet another stated that it is "first and foremost environmental reasons, then moral and ethical reasons, and then health reasons." Zoey is the only one that initially didn't go veg* for one of these reasons. She went vegan out of ``convenience.'' But she "started looking into animal rights and the environmental effects of vegetarianism later on."

Writing on veg* diets makes frequent use of these three categories. But no one yet has codified them. This categorization elucidates how those with unique diets feel the need to rationalize them to the outside world. Veg* diets are rare and unusual to most of the population. Thus, omnivores, as well as vegans and vegetarians, sometimes view following a veg* diet with no motive, or even for health reasons, as unsatisfying justification for rejecting many cultural norms.

The self-righteousness often attributed to veg* diets also helps explain this attitude \hyperlink{greenebaum}{(Greenebaum 2012)}. To dispel this smug aura, veg* people will employ the veg* trifecta in personal terms. They will assert that they are veg* because they feel \emph{personally} that it's the right thing to do. This is instead of claiming it's the right thing to do, thereby establishing some sort of external, universal moral code. Although this is a subtle rhetorical shift, it is in an attempt to deflect any backlash that they may face from this self-righteousness (\hyperlink{greenebaum}{Greenebaum 2012)}.

\subsection{Implicit Veg* Gendering}

When asked whether vegan and vegetarian diets are gendered, many women said no. One could not ``give any kind of gender\ldots,'' remarking that she ``{[}couldn't{]} decide whether it's feminine or masculine'' and ``hop{[}ing{]} it's universal.'' Another thought that ``it's a pretty gender-neutral thing.'' She does not ``associate being vegan as being {[}feminine or masculine{]}.'' Another woman also viewed veganism as a gender-neutral diet. Finally, one subject ``never even thought that was an issue.'' All other interviewees viewed the diets as masculine, discussed below. This was likely an attempt to save the masculinity of their men in this context by applying a masculine framework to the veg* lifestyle.

These results are in line with the data displayed in \hyperlink{table-2}{Table 2}. Every category other than ``Self-Reported Gender Expression'' is rather gender neutral, around a 3 on the scale of 1 to 5. All the men and women who took the survey supposedly act gender neutral and desire gender neutrality in their partner.

However, these statements and survey results contradict the transcripts. It is possible that subjects merely aspire to gender neutrality. It could also be that they are uneducated or too disinterested in the topic to have a solid opinion. I claim the former. As detailed below, most women have implicit ideas of masculinity and femininity as they interact with veg* diets.

\subsection{Strong-Kindness}

In general, the women I interviewed for this project perceived veg* diets as a masculine strength. They also value and desire more feminine traits from their men as well. The combination of these two traits is a concept I dub strong-kindness.

\subsubsection{Justice}

\paragraph{Kassity}

Kassity is a 35-year-old white, bisexual, ``non-monogamous'' woman. She grew up and continues to be upper-class. She had been vegan for approximately 7 years at the time of our interview and vegetarian for many years prior. It was an ethical consideration in her youth that grew to include environmental and health concerns. All this has culminated
into her work as an environmental and vegan activist.

She first met her now-husband via various random encounters, starting approximately 12 years ago. After dating four years, they became married in 2011. At that point, Kassity was vegetarian but interested in becoming vegan. Her husband was an omnivore at the time. But, as Kassity continued and completed her vegan transition, he began to follow suit. At the time of our interview, they had both been vegan together while raising a vegan family for about a year.

Kassity considers veganism as a strength, but in more gendered terms. She views vegans and vegan activists as ``real men, real women, {[}and{]} real humans'' that ``stand up for justice and integrity and honor,'' actions that are ``so sexy.'' She goes on to describe her ideal man as a ``righteous warrior," that is "\ldots due to reason, someone who's fighting on behalf of what's right in the world.'' Traditionally masculine concepts and imagery pervade her vocabulary.

Established forms of masculinity dictate men should be rational beings and impose that they should be aggressive. Kassity's ideal behavior differs in what ways these men display their aggression. According to traditional masculinity, men would be harming others with their aggression, whether verbal, emotional, or physical. For meat and masculinity, this often manifests in the killing of animals and the environment.

What Kassity finds attractive is the opposite. She likes this traditional masculine aggressiveness, describes it in such masculine terms, and finds it ``sexy.'' But she is also fond of her partner's veg* diet as it advocates for non-violence. For her, the pairing of aggression for a noble cause denotes justice.

\paragraph{Jade}

Jade is a 21-year-old Latino-American student who attends a private Californian university. She identifies as bisexual and had been vegetarian for three years at the time of our interview.

She grew up on a lower-class ranch in Texas. While discussing farm and meat culture, she mentioned how she became a vegetarian for environmental reasons. This seemed odd to me due to her background. Jade clarified she used to believe all farms treated animals the same way her family did on their property. Learning about factory farming forced her to reconsider.

Jade met her boyfriend during her senior year of high school whom she dated for three years. She started the discussion on vegetarianism after taking a class about the science behind environmental issues; she ended up transitioning first. In the meantime, her partner was invested in their conversations on the topic, eventually following suit a couple months later. He went vegetarian for environmental reasons too as well as health.

Jade views following a vegetarian diet as an element of a ``principled life.'' She explains that it indicates a person's ``amount of strength\ldots{}and that {[}they're{]} able to live that principled life in {[}their{]} actions rather than just in {[}their{]} words.'' In fact, it was this lack of a principled life that led to the end of her relationship with Jake. This was despite him ``being strong in making {[}the decision to go vegetarian{]} for himself.'' Because Jake brought up the idea first and followed through, Jade believes that it was a strength, especially because she's aware of the associated emasculation that veg* lifestyles harbor.

The way she explains the ``principled life" concept, a synonym for justice, demonstrates the underlying masculine ideals. The desire for others to live according to their values is fairly gender-neutral. But Jade implies masculinity. According to her, anything less of holding steadfast to your values is not strong. She admires the individuality of veg* diets and adhering to it despite the consequences. The fact that he was going against his background, ``a man\ldots{}raised in Texas,'' ``against a lot of social norms,'' and that it was something that he ``individually wanted,'' made him ``very strong in making that decision for himself." For men to expected to always live up to their values in every situation is yet another form of unrealistic strength standards imposed on men.

\paragraph{Overview}

These women view the values and politics associated with veg* diets as an attractive quality in a man. Some of these values transgress traditional masculinity. For example, concern for animal welfare is a feminine-coded concept because of its basis in emotion. These men may also have ``feelings in {[}their{]} politics,'' a very non-masculine trait.

However, these women often save the masculinity of their partner. Sometimes, it is with rhetoric. This takes a melodramatic form with Kassity's use of the ``righteous warrior.'' But it is most enacted while describing their strive towards justice as a strength, usually taking the form of always ``not being scared'' of living out one's values despite the context or circumstances. This pressure for men to be strong and to always hold to these justice-oriented values is indicative of the traditional masculine expectation to be unwavering in any situation.

When contrasting these experiences with one of the non-veg* interviewees, some differences emerge. Mai, for example, was wary of her vegan activist boyfriend's ``militant veganism.'' His extreme stance contributed to her leaving the lifestyle herself. This participant did not put as strong of an emphasis on justice as the others. Thus, she felt more uncomfortable in the relationship than did the other women who gave praise for their partner's justice-oriented values and actions.

\subsubsection{Strong Mind}

\paragraph{Mai}

Mai is a 22-year-old, Asian, heterosexual woman who works in the film industry. She became vegan due to ``peer pressure'' from her boyfriend at the time. He started the lifestyle after reading a book on the subject that Mai asked for me to redact. She had some investment in the environmental, health, and ethical issues he advocated for. Her partner's interest in veganism and his open-mindedness enamored her. She also cited her own curiosity and open-mindedness as personal reasons for starting the transition. But she mostly became and remained vegan to appease him.

To complicate matters, she had a strong ``urge'' to eat meat from the beginning of her vegan escapade. In fact, at times she would say, ```Hey, I'll do the dishes' and then {[}eat{]} the pieces of fat that my parents had thrown\ldots{}I was like, `\ldots it's not going to be any different if it goes in the trash or if it goes in my body.'\,''

Mai desires someone who is ``60\% logic, 40\% emotion'' to balance out her ``very emotional'' personality. She also ``desire{[}s{]} stability'' in her partner as well as a ``strong mind.'' She defines this state as when ``obstacles won't affect you'' with a ``motivation and positivity towards life.'' Despite the baggage from this relationship, she still admires this ``cool lifestyle {[}veganism{]} where your willpower is practiced everyday\ldots''

Mai emphasizes the strength aspect of strong-kindness in her interview. It could be that Mai's case is unique because of her self-described extremely emotional personality. But the effect is still the same for prospective partners: the expectation to have this unrealistic ``strong mind'' and unwavering stability.

\paragraph{Nare}

Nare is an Armenian-American, 28-year-old, heterosexual woman. Before becoming vegan, she had been a vegetarian for about six years for health reasons and animal rights. She had never enjoyed eating meat and found it ``depressing.''

She met her boyfriend at a wrap party on set. He was vegetarian at the time, transitioning to veganism. Despite his fondness for meat, he was veg* out of concern for animal welfare.

Nare values ``an ability to staying level-headed and rational and not very emotional\ldots'' However, she still admired and found necessary the times where her boyfriend ``would get really upset sometimes at\dots the things that matter\dots he would just need some time for himself\dots to write about it in his journal\dots he would get really distraught and empathetic with that event.''

In essence, Nare is on the rational side of things. Yet, she still is able to reconcile with the times when her boyfriend ``would get really upset.'' The issues he tended to get upset over were noble in nature, ``the things that matter,'' such as a ``shooting.'' She leaves personal emotional deliberation out of her assessment. Nare ascribes the strong mind concept to her partner by implying her ideal men would only become emotional over meaningful items that deserve it; to her, anything else would be indicative of weakness. She saves his masculinity while still applauding the emotion behind the issues that she cares about.

\paragraph{Zoey}

Zoey is a white, upper-middle class student who attends a university in California. She tried out veganism approximately four years ago out of convenience; meat preparation was tricky and veg* roommates made the transition easier. Because she ``got more into politics'' her sophomore year, her veganism is now more informed by animal rights and environmental issues.

She started dating her now boyfriend who was also vegan around that time. She describes him as ``anarcho-communist'' and a ``punk vegan.'' Along with having a love for animals, he believed that rejecting animal products was rejecting capitalism.

Zoey would ``get looks'' with her boyfriend, who sported ``spiked hair" and ``tats,'' which ``didn't bother him at all.'' She ``admired that quality of not being scared, of being yourself.'' Furthermore, she praised that ``he had a lot of feelings in his politics and for the pain and suffering that he saw.''

Being true to one's self is a trait desirable for any gender. However, Zoey crosses the line into unrealistic expectations for men by expressing how it appealed to her that opposition ``didn't bother him at all.'' Resistance affects everyone. Men especially are expected to be solid, unwavering rocks in relationships (\hyperlink{fisman}{Fisman et al. 2006}; \hyperlink{hitsch2}{Hitsch, Hortaçsu, and Ariely 2010}; \hyperlink{shackelford}{Shackelford et al. 2005}). This combines with her approval of the ``feelings in his politics'' to produce the strong-kindness concept. Her boyfriend should not show much emotion, especially for personal matters. But she still desires an emotional basis for his ``politics.''

\paragraph{Overview}

Other participants communicated similar desires in their interviews. One liked that her partner went ``against a lot of social norms,'' and that it was something that he ``individually wanted.'' He was ``very strong
in making that decision {[}to go vegan{]} for himself.''

The types of qualities these women present are not inherently negative. Rather, it are these women's expectations and how they describe them that recall traditional masculine ideals. They expect their partners to always be the rock in their relationship. This assumption can force men
to be an unnatural and unhealthy stoicism.

Furthermore, they desire an individualistic attitude in their ideal men. They like the ability of these men to remain staunch when rallying against societal norms despite the shame, insecurity, and other emotional trauma that can result. Their desires draw upon traditional tenants of emotional and mental strength ascribed to men.

\subsubsection{Discussion}

One of the most salient questions to arise out of this research is whether the results are generalizable to romantic relationships or if they only apply to veg* women. The fact that only two women in my sample were not veg* at some point advances this concern.

The results are at least generalizable to veg*, liberal, middle-class women. Beyond that, it's difficult to state broader implications. Most extant research bases itself around these groups, including Mycek's \hyperlink{mycek}{(2018)} study in which its sample comprised of white, middle-class men. Current research may at least point to the beginning of broader societal change.

\section{Conclusion}
\subsection{Limitations}

It is important to disclose that I am vegan and present as male. I did not inform participants that I am vegan during interviews. Some may have suspected that I am due to my knowledge of veg* issues and concepts. A couple participants happen to know beforehand that I was vegan. These traits could have implicitly or explicitly affected how they answered my questions. There may have been points when they did not want to hurt my feelings, for example. Or they might have felt uncomfortable talking to a man about various issues.

The sample recruited for this project is more representative than other studies on veg* diets. There were Latina and Asian subjects depicted and a lower-class-raised individual as well. Furthermore, there are interviewees from various parts of the United States and one who grew up outside America.

However, the majority of survey respondents and interviewees came from white and middle- to upper-class backgrounds. As mentioned earlier, this reflects the current demographics of American vegans and vegetarians \hyperlink{millum}{(Millum 2018}).

One of the more tenuous aspects of this study is the gender measures in the survey. I made an effort to consider the history of these measures and use multiple, recent versions for the survey. But one could still make an argument against the particular way I wrote the questions. They could even contest the inclusion of these questions at all. Some may point to the rather gender-neutral results as proof that these methods do not work.

For this reason and others, future researchers should seek a larger sample. For the purposes of an undergraduate project, the sample size is adequate. But to achieve greater representation and reliability, more interviews should be conducted. The survey suffers the most from this. At the time of this writing, there are a little over thirty respondents. With more time and resources, it may be possible to, for example, utilize a survey panel to get a representative picture of the American veg* population.

\hypertarget{future-research}{\subsection{Future Research}}

One of the unintended findings of my research was the link between sexuality and veg* diets. \hyperlink{table-1}{Table 1} reports that a full third of my sample identified as bisexual. Only about 3.5\% of Americans, however, identify as gay, lesbian, or bisexual \hyperlink{gates}{(Gates 2011)}.

I suspect that the reason for this disproportionality is due to the correlation between LGBT politics and the veg* politics. Both tend to lean liberal \hyperlink{reinhart}{(Reinhart 2019)}. There is likely a coincidental correlation rather than a biological or sociobiological cause. It would be difficult to test this and it is out of the scope of the current research.

Another option for future research is to do the same type of project and reverse the genders. In other words, men in relationships with veg* women would be asked how they view their partner's femininity. This study would serve to clarify aspects of masculinity and femininity like the current project. But this new work would center on constructions of femininity rather than masculinity. It would be an attempt to answer the inverse question of whether the femininity associated with veg* diets makes veg* women more attractive to heterosexual and bisexual vegan men.
