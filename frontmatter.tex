\pagecolor{white}
\def\contentsname{\empty}
\newgeometry{margin=1in, bottom=1.5in, top=1in}
\begin{multicols*}{2}
 \tableofcontents
 \addtocontents{toc}{\vspace*{-10.2em}}
\end{multicols*}

\newgeometry{margin=2.5in, top=1.5in, bottom=1.5in}

\section{Acknowledgements}

I'd like to thank:

\begin{itemize}
 \item
       My primary advisor, Edson Cruz. You have provided
       immense academic, professional, emotional, and personal support and
       inspiration for both this project and otherwise. I appreciate you so much.
\end{itemize}

\begin{itemize}
 \item
       Lynn Horton for reading drafts, professional and emotional support,
       and for those conversations after class about gender, neoliberalism,
       and other fun stuff.
 \item
       The London Interdisciplinary Center, The World Congress on
       Undergraduate Research, the Pacific Sociological Association, the
       American Men's Studies Association, and the American Sociological
       Association for seeing merit in my project by accepting me to present
       at their conferences.
 \item
       The Student Government Association, Wilkinson College, the Sociology
       Department, the Center for Undergraduate Excellence, the University
       Honors Department, Stephanie Takaragawa, Lisa Leitz, Ann Gordon,
       Carmichael Peters, Taryn Stroop, and Lisa Kendrick for believing
       enough in my project to send me all over the world to present my
       research. It means the world, professionally and personally.
 \item
       Sean Heim, Domenico Napoletani, Karen Knecht, Edson Cruz, Carmichael
       Peters, Lynn Horton, and Lisa Kendrick for their understanding and
       patience regarding my health the semester I was finishing this thesis.
 \item
       Catherine Bailey for your transformative guidance and support at a
       critical time in my life. You are probably the reason I'm doing
       sociology and for all the happiness it has brought me.
 \item
       My good friends and roommates Ben Bond, Samuel Reinhart, and Danny
       Cassee without whom home for me wouldn't exist. Ben, thanks for going
       vegan with me in the caf and for being my best friend.
 \item
       Sierra Segal for your love, support, and much-needed boosting of my
       self-esteem.
 \item
       Obligatory parents for their support. Thank you for coming to that
       weird veg* conference with me, mom.
\end{itemize}

\section{Abstract}

As more men attempt vegan or vegetarian (collectively referred to as veg*) lifestyles, the historic link between meat and masculinity has become more pronounced. Women tend to gravitate toward several traits associated with veg* diets (e.g., compassion and health). However, the emasculation associated with these diets may also repel them.

This project utilized a mixed-methods analysis. It uses survey data (\emph{N} = 28) and in-depth, one-on-one interviews with six women who have dated veg* men. I recruited subjects using convenience sampling from personal social networks and veg* conferences. The present study explores whether heterosexual and bisexual women are attracted to the masculinity of vegan and vegetarian men more than their omnivore counterparts.

Findings indicate that most women tend to view veg* diets as a masculine strength and an indicator of kindness, a concept I dub strong-kindness. With strong-kindness, this version of ideal masculinity moves a little more into the feminine side while still remaining firmly within the masculine part of the binary. Utilizing aspects of veg* lifestyles might improve the health of all as well as the environment. Removing the stigma associated with these lifestyles may help improve the lives of
all individuals regardless of gender identity.

\raggedbottom
\pagebreak

\section{Introduction}

There is no doubt a connection between the consumption of meat and masculinity. It appears in books, articles, and as a trope in the media: from the rugged cowboy figure killing and eating his own game to recent ads depicting men dismissing tofu for a double Whopper \hyperlink{sobal}{(Sobal 2006).} And this phenomenon is not limited to the Western hemisphere. It exists in places from Argentina to Japan (\hyperlink{delessio-parson}{DeLessio-Parson 2017}; \hyperlink{morioka}{Morioka 2013}).

Many call on biology and human evolution to explain men's supposed natural affliction for meat. Researchers draw upon men's stronger physique and reference the adaptations humans have gone under to process meat better than our primate ancestors. All this is to argue that nature has designed men to eat meat (\hyperlink{leroy}{Leroy and Praet 2015}; such as discussed by \hyperlink{zink}{Zink and Lieberman 2016}). By extension, some argue that women desire these masculine traits as well (\hyperlink{sell}{Sell, Lukazsweski, and Townsley 2017}).

More and more men recently, however, are beginning to adopt a vegan or vegetarian lifestyle (\hyperlink{mycek}{Mycek 2018}). This way of life goes against these supposed social and biological norms. There are multiple studies that examine veg* men's masculinity. Members of all diets tend to view veg* men as more righteous. But they still consider them less masculine than their carnivorous counterparts for choosing to forego meat (\hyperlink{ruby}{Ruby and Heine 2011}; \hyperlink{thomas}{Thomas 2015}). Interviews with meat-eating men reveal that an often cited explanation for consuming meat is to maintain masculinity \hyperlink{newcombe}{(Newcombe et al. 2012)}. Talking with veg* men also uncover the masculine ways in which they discuss their veg* diets to conserve their masculinity \hyperlink{mycek}{(Mycek 2018)}.

There are existing discussions on the role of meat in heterosexual, romantic relationships where either both partners or the man is omnivorous. But few have explored when the roles are reversed \hyperlink{sobal}{(Sobal 2006)}. This has lead to my current research which asks if heterosexual and bisexual women are attracted to vegan and vegetarian men more than their omnivore counterparts.

\restoregeometry
